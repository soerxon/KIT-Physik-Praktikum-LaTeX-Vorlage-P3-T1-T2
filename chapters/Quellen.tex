\newpage
\addcontentsline{toc}{chapter}{Quellen}
\renewcommand{\refname}{Quellen}   
\renewcommand{\bibname}{Quellen} % Lässt die eigene Überschrift des Literaturverzeichnisses verschwinden. Es wird nun "Quellen" angezeigt.
\bibliography{include/Quellen.bib}
\bibliographystyle{alphadin}

% Dieser Befehl kann genutzt werden um hinter einem Text einen Verweis auf die Quelle in der Quellenangabe erscheinen zu lassen.
%\cite{Example1}
% Dieser Befehl kann genutzt werden um Quellenangaben in den Quellen aufzulisten, auch wenn im Text nicht darauf verwiesen wird.
%\nocite{*}